\part{Composition Event}

\chapter{Overview}


Composition Event(复合事件)是DOM3级事件中新添加的一类事件类型,用于处理IME的输入序列。

IME(Input Method Editor,输入法编辑器)可以让用户输入在物理键盘上找不到的字符,复合事件就是针对检测和处理这种输入而设计的。

所有的拉丁字母都可以通过物理键盘输入,因此复合事件很少被拉丁系语言输入的开发者了解,而且在开发中使用到复合事件类型的情况比较少见。

IME复合系统的工作原理是缓存用户的键盘输入,直到一个字符被选中后才确定输入,缓存的键盘输入会暂时展示在输入框中,并不会真正被插入到DOM中。

如果在复合事件的过程中改变了输入框的值(例如切换了输入法或者直接按下Enter键),那么复合事件就会提前结束,同时缓存的键盘输入值将会被插入到输入框中。

复合事件类型包含以下几种事件:

\begin{compactitem}
\item compositionstart:在IME的文本复合系统打开时触发;
\item compositionend:在IME的文本复合系统关闭(即用户选中了字符并确定输入)时触发,表示返回正常键盘的输入状态;
\item compositionupdate:在compositionstart事件触发后、compositionend事件触发前的这段时间内,每次向输入字段中进行输入时均会触发。
\end{compactitem}

注意,input事件将在复合事件后触发。

在实际情况中,复合事件的兼容性并不好,因此在使用复合事件处理input相关问题时,仍然需要慎重。

\section{compositionstart}


\section{compositionend}


\section{compositionupdate}






