\part{G2}

\chapter{Overview}


G2 (The Grammar Of Graphics) 是一个由纯 JavaScript 编写、强大的语义化图表生成工具,它提供了一整套图形语法,可以让用户通过简单的语法搭建出无数种图表,并且集成了大量的统计工具,支持多种坐标系绘制,可以让用户自由地定制图表,是为大数据时代而准备的强大的可视化工具。

g2支持通过将脚本下载到本地也支持直接引入在线资源。

\begin{lstlisting}[language=HTML]
<!-- 引入在线资源 -->
<script src="https://a.alipayobjects.com/g/datavis/g2/2.2.5/g2.js"></script>
<!-- 引入本地脚本 -->
<script src="./g2.js"></script>
\end{lstlisting}

在 G2 引入页面后,就已经做好了创建第一个图表的准备了。例如,下面是以一个基础的柱状图为例开始我们的第一个图表创建。

在页面的body部分创建一个 div,并指定必须的属性 id:

\begin{lstlisting}[language=HTML]
<div id="c1"></div>
\end{lstlisting}

在创建 div 容器后就可以进行简单的图表绘制,具体步骤如下:

\begin{compactenum}
\item 创建 Chart 图表对象,指定图表所在的容器 ID、指定图表的宽高、边距等信息;
\item 载入图表数据源;
\item 使用图形语法进行图表的绘制;
\item 渲染图表。
\end{compactenum}


如果使用 <script></script>包裹g2图表的代码,那么可以放在页面代码的任意位置(最好的做法是放在<body>之前)。


\begin{lstlisting}[language=HTML]
var data = [
  {genre: 'Sports', sold: 275},
  {genre: 'Strategy', sold: 115},
  {genre: 'Action', sold: 120},
  {genre: 'Shooter'复制代码, sold: 350},
  {genre: 'Other', sold: 150},
];

var chart = new G2.Chart({
   id: 'c1',
   width: 600,
   height: 300
});

chart.source(data, {
   genre: {}
});
\end{lstlisting}

\section{NPM}


\begin{lstlisting}[language=bash]
$ npm install g2 --save
\end{lstlisting}

在成功安装g2之后就可以使用 import 或 require 进行引用,例如:

\begin{lstlisting}[language=JavaScript]
var G2 = require('g2');
var chart = new G2.Chart({
    id: 'c1',
    width: 600,
    height: 300
});
\end{lstlisting}



\begin{lstlisting}[language=bash]

\end{lstlisting}




\begin{lstlisting}[language=bash]

\end{lstlisting}