\part{Vue指令}


\chapter{Overview}





\section{v-if和v-show}

在切换v-if模块时,Vue.js有一个局部编译/卸载工程,因为v-if中的模板可能包括数据绑定或子组件。

\begin{compactitem}
\item v-if是真实的条件渲染,可以确保条件块在切换时合适地销毁和重建条件块内的事件监听器和子组件。
\item v-show是切换内联CSS样式,元素始终被编译并保留,只是简单地基于CSS进行切换。
\end{compactitem}

一般来说,v-if有更高的切换消耗,v-show有更高的初始渲染消耗。

v-if是惰性的——如果初始渲染条件为false,那么不渲染,在渲染条件第一次变为true时才开始局部编译(并缓存起来)。

\begin{compactitem}
\item 如果需要频繁的切换,则使用v-show较好。
\item 如果在运行时提交不会频繁切换,则使用v-if较好。
\end{compactitem}

\section{v-el和v-ref}

v-el为DOM元素注册一个索引,可以通过所属实例的\$els访问这个元素。

\begin{compactitem}
\item 通过\texttt{v-el:some-el}设置\texttt{this.\$els.someEl}。


\begin{lstlisting}[language=JavaScript]
<span v-el:msg>hello</span>
<span v-el:other-msg>world</span>
\end{lstlisting}

\item 通过\texttt{this.\$els}获取相应的DOM元素。

\begin{lstlisting}[language=JavaScript]
this.$els.msg.textContent; // -> hello
this.$els.otherMsg.textContent; // -> world
\end{lstlisting}

\end{compactitem}

v-ref在父组件上注册一个子组件的索引来方便访问。

v-ref不需要表达式,必须提供参数id,可以通过父组件的\$refs对象访问子组件。

\begin{compactitem}
\item 如果v-ref和v-for一起使用时,注册的值将是一个数组,包含所有的子组件,对应于绑定数组。
\item 如果v-ref使用在一个对象上,注册的值将是一个对象,包含所有的子组件,对应于绑定对象。
\end{compactitem}

HTML不区分大小写,camelCase风格的名字(例如v-ref:someRef)将全部转换为小写,可以使用v-ref:some-ref设置this.\$els.someRef。










\begin{lstlisting}[language=JavaScript]

\end{lstlisting}




\begin{lstlisting}[language=JavaScript]

\end{lstlisting}




\begin{lstlisting}[language=JavaScript]

\end{lstlisting}




\begin{lstlisting}[language=JavaScript]

\end{lstlisting}




\begin{lstlisting}[language=JavaScript]

\end{lstlisting}




\begin{lstlisting}[language=JavaScript]

\end{lstlisting}


\chapter{Build-in Directive}

\section{v-text}


v-text可以更新元素的 textContent。如果要更新部分的 textContent ,需要使用 \{\{ Mustache \}\} 插值。

实际上,\texttt{\{\{Mustache\}\}}在内部也被编译为textNode的一个v-text指令。

\subsection{类型}

\begin{compactitem}
\item 类型:string
\end{compactitem}


\begin{lstlisting}[language=JavaScript]
<span v-text="msg"></span>
<!-- 和下面一样 -->
<span>{{ msg }}</span>
\end{lstlisting}



\begin{lstlisting}[language=JavaScript]

\end{lstlisting}




\begin{lstlisting}[language=JavaScript]

\end{lstlisting}




\begin{lstlisting}[language=JavaScript]

\end{lstlisting}




\begin{lstlisting}[language=JavaScript]

\end{lstlisting}




\begin{lstlisting}[language=JavaScript]

\end{lstlisting}




\begin{lstlisting}[language=JavaScript]

\end{lstlisting}




\begin{lstlisting}[language=JavaScript]

\end{lstlisting}

\section{v-html}

v-html可以更新元素的 innerHTML 。

注意,内容按普通 HTML 插入,不会作为 Vue 模板进行编译,数据绑定被忽略。

\begin{compactitem}
\item 如果试图使用 v-html 组合模板,可以重新考虑是否通过使用组件(Component)来替代。
\item 如果需要复用模板片段,则应当使用partials。
\end{compactitem}

实际上,\texttt{\{\{\{Mustache\}\}\}}在内部也被编译为锚节点上的一个v-html指令。

不建议直接动态渲染任何HTML片段,容易导致XSS攻击。

\subsection{类型}

\begin{compactitem}
\item 类型:string
\end{compactitem}

在网站上动态渲染任意 HTML 是非常危险的,因为容易导致 XSS 攻击。

仅仅只在可信内容上使用 v-html,永远不要用在用户提交的内容上。


\begin{lstlisting}[language=JavaScript]
<div v-html="html"></div>
<!-- 和下面的相同 -->
<div>{{{ html }}}</div>
\end{lstlisting}



\begin{lstlisting}[language=JavaScript]

\end{lstlisting}




\begin{lstlisting}[language=JavaScript]

\end{lstlisting}




\begin{lstlisting}[language=JavaScript]

\end{lstlisting}




\begin{lstlisting}[language=JavaScript]

\end{lstlisting}




\begin{lstlisting}[language=JavaScript]

\end{lstlisting}




\begin{lstlisting}[language=JavaScript]

\end{lstlisting}




\begin{lstlisting}[language=JavaScript]

\end{lstlisting}

\section{v-show}

v-show可以根据表达式的真假值来切换元素的display CSS 属性,从而实现显示或隐藏HTML元素的效果。

\begin{compactitem}
\item 当v-show表达式赋值为false时,HTML元素上增加一个内联样式\texttt{style="display: none"}
\item 当v-show表达式赋值为true时,HTML元素的内联样式更新为原始样式。
\end{compactitem}

\subsection{类型}

\begin{compactitem}
\item 类型:any
\end{compactitem}

当条件变化时该指令触发过渡效果。

\subsection{DOM}

\subsubsection{元素}


\begin{lstlisting}[language=JavaScript]
<html>
  <head>
    <script src="/path/to/vue.js"></script>
  </head>
  <body>
  <input type="text" v-model="message" placeholder="edit me">
  <div id="example">
    <p v-show="greeting">Hello</p>
  </div>
  </body>
  <script>
  var exampleVM = new Vue({
    el: '#example',
    data: {
      greeting: false
    }
  })
  </script>
</html>
\end{lstlisting}

\subsubsection{模板}

v-show不支持<template>语法。


\begin{lstlisting}[language=JavaScript]

\end{lstlisting}



\begin{lstlisting}[language=JavaScript]

\end{lstlisting}




\begin{lstlisting}[language=JavaScript]

\end{lstlisting}




\begin{lstlisting}[language=JavaScript]

\end{lstlisting}




\begin{lstlisting}[language=JavaScript]

\end{lstlisting}




\begin{lstlisting}[language=JavaScript]

\end{lstlisting}




\begin{lstlisting}[language=JavaScript]

\end{lstlisting}




\begin{lstlisting}[language=JavaScript]

\end{lstlisting}

\section{v-if}

v-if可以根据表达式的值的真假条件渲染元素。

\begin{compactitem}
\item 在v-if切换时,元素及它的数据绑定/组件被销毁并重建。
\item 在v-if切换时,如果元素是 <template> ,将提取出它的内容作为条件块。
\end{compactitem}

当条件变化时,v-if指令触发过渡效果。

\subsection{类型}


\begin{compactitem}
\item 类型:any
\end{compactitem}



\subsection{DOM}

v-if指令可以完全根据表达式的值在DOM中生成或移除一个元素(Element)。

\begin{compactitem}
\item 如果v-if表达式赋值为false,那么对应的元素就会从DOM树中被移除;
\item 如果v-if表达式赋值为true,那么对应的元素的一个克隆就会被重新插入到DOM树中。
\end{compactitem}


\begin{lstlisting}[language=JavaScript]
<html>
  <head>
    <script src="/path/to/vue.js"></script>
  </head>
  <body>
  <div id="example">
    <p v-if="greeting">Hello</p>
  </div>
  </body>
  <script>
  var exampleVM = new Vue({
    el: '#example',
    data: {
      greeting: false
    }
  })
  </script>
</html>
\end{lstlisting}

\subsubsection{元素}

v-if需要添加到元素上才能使用。

\begin{lstlisting}[language=JavaScript]

\end{lstlisting}


\subsubsection{模板}

v-if切换多个元素时,可以把<template>元素作为包装元素,并在template上使用v-if,最终的渲染结果不会包含<template>。


\begin{lstlisting}[language=JavaScript]
<template v-if="ok">
  <h1>Title</h1>
  <p>Paragraph 1</p>
  <p>Paragraph 2</p>
</template>
\end{lstlisting}




\begin{lstlisting}[language=JavaScript]

\end{lstlisting}




\begin{lstlisting}[language=JavaScript]

\end{lstlisting}




\begin{lstlisting}[language=JavaScript]

\end{lstlisting}




\begin{lstlisting}[language=JavaScript]

\end{lstlisting}




\begin{lstlisting}[language=JavaScript]

\end{lstlisting}

\section{v-else}

v-else可以为 v-if 或者 v-else-if 添加 “else 块”,而且v-else必须跟着v-if或v-show。


\subsection{类型}

\begin{compactitem}
\item 类型:any
\end{compactitem}


\begin{lstlisting}[language=JavaScript]
<div v-if="Math.random() > 0.5">
  Now you see me
</div>
<div v-else>
  Now you don't
</div>
\end{lstlisting}

\begin{compactitem}
\item 不需要表达式
\item 前一兄弟元素必须有 v-if 或 v-else-if
\end{compactitem}


\subsection{DOM}

\subsubsection{元素}


\begin{lstlisting}[language=JavaScript]
<html>
  <head>
    <script src="/path/to/vue.js"></script>
  </head>
  <body>
  <div id="example">
    <p v-if="ok">Hello</p>
    <p v-else>Hi</p>
  </div>
  </body>
  <script>
  var exampleVM = new Vue({
    el: '#example',
    data: {
      ok: false
    }
  })
  </script>
</html>
\end{lstlisting}


\subsubsection{组件}

v-else和v-show的的优先级问题导致v-show不适合用在组件上。

\begin{compactitem}
\item v-else不适合用在组件上

\begin{lstlisting}[language=JavaScript]
<!-- v-else不适合用在组件上 -->
<custom-component v-show="condition"></custom-component>
<p v-else>这可能也是一个组件</p>
\end{lstlisting}

\item v-show代替v-else用在组件上

\begin{lstlisting}[language=JavaScript]
<!-- v-show代替v-else -->
<custom-component v-show="condition"></custom-component>
<p v-show="!!condition">这可能也是一个组件</p>
\end{lstlisting}

\end{compactitem}





\begin{lstlisting}[language=JavaScript]

\end{lstlisting}




\begin{lstlisting}[language=JavaScript]

\end{lstlisting}




\begin{lstlisting}[language=JavaScript]

\end{lstlisting}




\begin{lstlisting}[language=JavaScript]

\end{lstlisting}




\begin{lstlisting}[language=JavaScript]

\end{lstlisting}




\begin{lstlisting}[language=JavaScript]

\end{lstlisting}


\section{v-else-if}


v-else-if可以为v-if表示“else if块”,支持链式使用。

\subsection{类型}

\begin{compactitem}
\item 类型:any
\end{compactitem}





\begin{lstlisting}[language=JavaScript]
<div v-if="type === 'A'">
   A
</div>
<div v-else-if="type ==='B'">
   B
</div>
<div v-else-if="type === 'C'">
   C
</div>
<div v-else>
   Not A/B/C
</div>
\end{lstlisting}

\begin{compactitem}
\item 前一个同级元素必须具有v-if或v-else-if。
\end{compactitem}

\begin{lstlisting}[language=JavaScript]

\end{lstlisting}




\begin{lstlisting}[language=JavaScript]

\end{lstlisting}




\begin{lstlisting}[language=JavaScript]

\end{lstlisting}




\begin{lstlisting}[language=JavaScript]

\end{lstlisting}




\begin{lstlisting}[language=JavaScript]

\end{lstlisting}




\begin{lstlisting}[language=JavaScript]

\end{lstlisting}




\begin{lstlisting}[language=JavaScript]

\end{lstlisting}

\section{v-for}

v-for可以基于源数据多次渲染元素或模板块,也可以使用\$index显示对应的数组索引。


\begin{lstlisting}[language=JavaScript]
<html>
  <head>
    <script src="/path/to/vue.js"></script>
  </head>
  <body id="example">
  <ul id="demo">
    <li v-for="item in items">
      {{ item.$index }} - {{ parentMessage }} {{ item.msg }}
    </li>
  </ul>
  </body>
  <script>
  var exampleVM = new Vue({
    el: '#demo',
    data: {
      parentMessage: '滴滴',
      items: [
        { msg: '顺风车' },
        { msg: '专车' }
      ]
    }
  })
  </script>
</html>
\end{lstlisting}

v-for可以和Vue.js内置的过滤器或排序器一起使用。

\begin{compactenum}
\item filterBy

\begin{compactitem}
\item 语法:\texttt{filterBy searchKey [in dataKey ...]}
\item 用法:

\begin{lstlisting}[language=JavaScript]
<input v-model="searchText">
<ul>
  <li v-for="user in users | filterBy searchText in 'name' ">{{user.name}}</li>
</ul>

data: {
  users: [
     {
       name: '快车',
       tag: '1'
     },
     {
       name: '出租车',
       tag: '2'
     },
     {
       name: '专车',
       tag: '4'
     }
  ]
}
\end{lstlisting}

\end{compactitem}

\item orderBy


\begin{compactitem}
\item 语法:\texttt{orderBy sortKey [reverseKey]}
\item 用法:

\begin{lstlisting}[language=JavaScript]
<body id="example">
  <ul>
    <li v-fo="user in users | orderBy field reverse">{{user.name}}</li>
  </ul>
</body>
<script>
var exampleVM = new Vue({
  el: '#example',
  data: {
    field: 'tag',
    reverse: false,
    users: [
       {
         name: '快车',
         tag: 1
       },
       {
         name: '出租车',
         tag: 2
       },
       {
         name: '专车',
         tag: 4
       }
    ]
  }
})
</script>
\end{lstlisting}

\end{compactitem}

\end{compactenum}


\subsection{类型}

\begin{compactitem}
\item 类型:Array|Object|number|string
\end{compactitem}

v-for指令的值必须使用特定语法\texttt{alias in expression}来为当前遍历的元素提供别名:


\begin{lstlisting}[language=JavaScript]
<div v-for="item in items">
  {{ item.text }}
</div>
\end{lstlisting}

v-for支持遍历数组、对象、数字和字符串。




v-for支持\texttt{item in/of items}的形式,其中of分隔符更接近JavaScript的遍历器语法。

\subsubsection{Object}

v-for来遍历一个对象时,每一个重复的实例都将有一个特殊的属性\$key或者给对象的键值提供一个别名。



\begin{lstlisting}[language=JavaScript]
<html>
  <head>
    <script src="/path/to/vue.js"></script>
  </head>
  <body id="example">
  <ul id="demo">
    <li v-for="value in primitiveValues">{{ $key}}:{{$value}}</li>
    <li>=====</li>
    <li v-for="(key, item) in objectValues">{{key}}:{{item.msg}}</li>
  </ul>
  </body>
  <script>
  var exampleVM = new Vue({
    el: '#demo',
    data:{
      primitiveValues: {
        FirstName: 'Lei',
        LastName: 'Li',
        Age: 34
      },
      objectValues: {
        one: {
           msg: 'Hello'
        },
        two: {
           msg: 'World'
        }
      }
    }
  })
  </script>
</html>
\end{lstlisting}

ES5无法检测到新属性添加到一个对象上或者在对象中删除,Vue.js增加了三种方法\$add、\$set和\$delete来添加和删除属性,同时触发视图更新。

\begin{compactitem}
\item \texttt{\$add(key,value)}
\item \texttt{\$set(key,value)}
\item \texttt{\$delete(key)}
\end{compactitem}



\subsubsection{Integer}

v-for支持整数。


\begin{lstlisting}[language=JavaScript]
<div id="range">
  <div v-for="n in 10">{{$index}}</div>
</div>
\end{lstlisting}

\subsection{别名}

v-for需要特别的别名,形式为\texttt{item in items}。例如,这里的items为数据数组,item就是当前数组元素的别名。

v-for在开始时会对传入的表达式进行语法分析,如果不是\texttt{item in/of items}的形式就会给出警告信息。


\subsubsection{index}



v-for支持为数组索引指定别名(或者用于对象的键):



\begin{lstlisting}[language=JavaScript]
<div v-for="(item, index) in items"></div>
<div v-for="(val,key) in object"></div>
<div v-for="(val,key,index) in object"></div>
\end{lstlisting}

上述示例可以改写为:


\begin{lstlisting}[language=JavaScript]
<html>
  <head>
    <script src="/path/to/vue.js"></script>
  </head>
  <body id="example">
  <ul id="demo">
    <li v-for="(item,index) in items">
      {{ index }} - {{ parentMessage }} {{ item.msg }}
    </li>
  </ul>
  </body>
  <script>
  var exampleVM = new Vue({
    el: '#demo',
    data: {
      parentMessage: '滴滴',
      items: [
        { msg: '顺风车' },
        { msg: '专车' }
      ]
    }
  })
  </script>
</html>
\end{lstlisting}

\subsubsection{key}


v-for默认行为将尝试不改变整体,只是替换元素。如果需要让v-for重新排序的元素,需要提供一个key的特殊属性,例如:


\begin{lstlisting}[language=JavaScript]
<div v-for="item in items" :key="item.id">
  {{ item.text }}
</div>
\end{lstlisting}



\subsection{作用域}

v-for会产生一个特殊的作用域,和AngularJS的隔离作用域类似,需要明确指定props属性来传递数据,否则在组件内将获取不到数据。例如,对于组件内的<p>标签,可以使用<slot>:


\begin{lstlisting}[language=JavaScript]
<my-item v-for="item in items" :item="item" :index="$index">
  <p>{{ item.text }}</p>
</my-item>
\end{lstlisting}

\subsection{包装方法}

Vue.js包装了被观察数据的变异方法来检测数据变动,这样当数据数据出现变动时就可以触发视图更新。


Vue.js包装的数据变动检测方法包括:

\begin{compactitem}
\item push()
\item pop()
\item shift()
\item unshift()
\item splice()
\item sort()
\item reverse()
\end{compactitem}


Vue.js在重写上述方法后触发了一次notify。


\subsubsection{\$set}

Vue.js还增加了两个方法来观测变化,分别是\$set和\$remove。

Vue.js要求尽量避免直接设置数据绑定的数组元素,否则这些变化不会被Vue.js检测到,也就不会更新视图渲染,\$set方法可以解决这个问题。




\begin{lstlisting}[language=JavaScript]
// same as demo.items[0] = ... but trigger view update
demo.items.$set(0, {childMsg: 'Changed!'})
\end{lstlisting}


\subsubsection{\$remove}

\$remove是splice的语法糖,可以用来从目标数组中查找并删除元素。

\begin{compactitem}
\item 不使用\$remove执行查找并删除元素

\begin{lstlisting}[language=JavaScript]
var index = this.items.indexOf(item)
if (index !== -1) {
  this.items.splice(index,1)
}
\end{lstlisting}

\item 使用\$remove查找并删除元素

\begin{lstlisting}[language=JavaScript]
demo.items.$remove(item)
\end{lstlisting}

\end{compactitem}


Vue.js支持使用filter、concat和slice方法等来处理数组,返回的数组将会是一个不同的实例,而且可以使用新的数组替换原来的数组。


\begin{lstlisting}[language=JavaScript]
demo.items = demo.items.filter(function(item) {
   return item.childMsg.match(/Hello/)
})
\end{lstlisting}


在某些情况下可能会需要全新对象(例如通过API调用创建的对象)来替换数组。

默认情况下,v-for通过数据对象的特征来决定对已有作用域和DOM元素的复用程度,可能导致重新渲染整个列表。不过,如果每个对象都有一个唯一的ID属性,那么就可以使用track-by特性来提示Vue.js来尽可能地复用已有的实例。

\begin{lstlisting}[language=JavaScript]
{
   items: [
     { _uid: '989f86d', ... },
     { _uid: '78689df', ... }
   ]
}
\end{lstlisting}

针对上述情况,可以给Vue.js发出提示,例如:

\begin{lstlisting}[language=JavaScript]
<div v-for="item in items" track-by="_uid">
  <!-- ... -->
</div>
\end{lstlisting}

Vue.js在替换数组时,如果遇到一个包含有\_uid等于已有值的新对象,就会复用对应的已有对象的作用域和DOM元素。

如果没有唯一的键可以追踪,Vue.js可以使用\texttt{track-by="\$index"}来强制v-for进入原位更新模式——片段不会被移动,而是简单地以对应索引的新值刷新,同时也可以用来处理数据数组中重复的值。

track-by在让数据替换高效的同时也需要付出一定的代价,这时DOM节点不再映射数组元素顺序的改变,不能同步临时状态(例如<input>元素的值)以及组件的私有状态。


如果v-for块包含<input>元素或子组件,必须要小心使用\texttt{track-by="\$index"}。

JavaScript的限制使得Vue.js不能检测下面数组的变化:

\begin{compactitem}
\item 直接用索引设置元素,例如\texttt{vm.items[0] = \{\}}

为了解决这个问题,Vue.js扩展观察数组来实现了\$set方法:

\begin{lstlisting}[language=JavaScript]
vm.items.$set(0, { childMsg: 'Changed' })
\end{lstlisting}

\item 修改数组的长度,例如\texttt{vm.items.length = 0}

\begin{lstlisting}[language=JavaScript]
vm.items = []
\end{lstlisting}

\end{compactitem}

如果需要重复一个包含多个DOM元素的块,可以使用<template>标签来包装重复片段,这里的<template>标签只是作为一个语义包装器。

\begin{lstlisting}[language=JavaScript]
<ul>
  <template v-for="list in lists">
    <li>{{list.msg}}</li>
    <li class="divider"></li>
  </template>
</ul>
\end{lstlisting}


\section{v-on}

v-on可以绑定事件监听器,v-on绑定的事件类型由参数指定。




\begin{lstlisting}[language=JavaScript]
<!-- 方法处理器 -->
<button v-on:click="doThis"></button>
<!-- 缩写 -->
<button @click="doThis"></button>
\end{lstlisting}

表达式可以是一个方法的名字或一个内联语句,如果没有修饰符,也可以省略。

\subsection{类型}

\begin{compactitem}
\item 缩写:@
\item 类型:Function|Inline Statement
\item 参数:event(必选)
\end{compactitem}


如果使用内联语句,语句可以访问一个 \$event 属性。例如,\texttt{v-on:click="handle('ok', \$event)"}。

\begin{lstlisting}[language=JavaScript]
<!-- 内联语句 -->
<button v-on:click="doThat('hello',$event)"></button>
\end{lstlisting}

\begin{compactitem}
\item v-on用在普通元素上时,只能监听 原生 DOM 事件。

在监听原生 DOM 事件时,如果只定义一个数,DOM event为事件的唯一参数。如果在内联语句中处理器中访问原生DOM事件,可以用特殊变量\$event把事件传入方法。


\begin{lstlisting}[language=JavaScript]
<!-- 方法处理器 -->
<button v-on:click="doThis"></button>

<!-- 内联语句 -->
<button v-on:click="doThat('hello',$event)"></button>
\end{lstlisting}


\item v-on用在自定义元素组件上时,也可以监听子组件触发的自定义事件。

自定义事件的内联语句可以访问一个\$arguments属性,其本质是一个数组,包含了传给子组件的\$emit回调的参数。



\end{compactitem}


\begin{lstlisting}[language=JavaScript]
<!-- 方法处理器 -->
<button v-on:click="doThis"></button>

<!-- 内联语句 -->
<button v-on:click="doThat('hello',$event)"></button>

<!-- 缩写 -->
<button @click="doThis"></button>

<!-- 停止冒泡 -->
<button @click.stop="doThis"></button>

<!-- 阻止默认行为 -->
<button @click.prevent="doThis"></button>

<!-- 阻止默认行为,没有表达式 -->
<form @submit.prevent></form>

<!-- 串联修饰符 -->
<button @click.stop.prevent="doThis"></button>

<!-- 键修饰符,键别名 -->
<input @keyup.enter="onEnter">

<!-- 键修饰符,键代码 -->
<input @keyup.13="onEnter''>
\end{lstlisting}

在子组件上监听自定义事件(当子组件触发 “my-event” 时将调用事件处理器):

\begin{lstlisting}[language=JavaScript]
<my-component @my-event="handleThis''></my-component>

<!-- 内联语句 -->
<my-component @my-event="handleThis(123,$event)"></my-component>

<!-- 组件中的原生事件 -->
<my-component @click.native="onClick"></my-component>
\end{lstlisting}


\subsection{修饰符}


v-on不仅支持支持参数,也支持修饰符,其中v-on支持的修饰符如下:


\begin{compactitem}
\item .stop:调用event.stopPropagation()
\item .prevent:调用event.preventDefault()
\item .capture:添加事件监听器时使用capture模式
\item .self:只当事件是从监听器绑定的元素本身触发时才触发回调
\item .\{keyCode|keyAlias\}:只有当事件是从监听器绑定的元素本身触发时才触发回调
\item .native:监听组件根元素的原生事件
\end{compactitem}

表达式可以是一个方法的名字或一个内联语句,如果没有修饰符也可以省略。






\begin{lstlisting}[language=JavaScript]

\end{lstlisting}




\begin{lstlisting}[language=JavaScript]

\end{lstlisting}




\begin{lstlisting}[language=JavaScript]

\end{lstlisting}




\begin{lstlisting}[language=JavaScript]

\end{lstlisting}




\begin{lstlisting}[language=JavaScript]

\end{lstlisting}




\begin{lstlisting}[language=JavaScript]

\end{lstlisting}

\section{v-bind}

v-bind可以动态地绑定一个或多个特性,或一个组件 prop 到表达式。


\subsection{类型}

\begin{compactitem}
\item 缩写:\texttt{:}
\item 类型:any(with argument)|Object(without argument)
\item 参数:attrOrProp(可选)
\end{compactitem}



\begin{lstlisting}[language=JavaScript]
<!-- 绑定一个属性 -->
<img v-bind:src="imageSrc">

<!-- 缩写 -->
<img :src="imageSrc">
\end{lstlisting}


\begin{compactitem}
\item v-bind在绑定 class 或 style 特性时,支持其它类型的值(例如数组或对象)。


\begin{lstlisting}[language=JavaScript]
<!-- class绑定 -->
<div :class="{ red: isRed }"></div>
<div :class="[classA, classB]"></div>
<div :class="[classA, { classB: isB, classC: isC }]"></div>

<!-- style绑定 -->
<div :style="{ fontSize: size + 'px' }"></div>
<div :style="[styleObjectA, styleObjectB]"></div>
\end{lstlisting}

\item v-bind在绑定 prop 时,prop 必须在子组件中声明。可以用修饰符指定不同的绑定类型。

\begin{lstlisting}[language=JavaScript]
<!-- 绑定一个有属性的对象 -->
<div v-bind="{ id: someProp, 'other-attr': otherProp }"></div>

<!-- 通过prop修饰符绑定DOM属性 -->
<div v-bind:text-content.prop="text"></div>

<!-- prop绑定,prop必须在my-component中声明 -->
<my-component :prop="someThing"></my-component>
\end{lstlisting}

\end{compactitem}

没有参数时,v-bind可以绑定到一个对象。

\begin{lstlisting}[language=JavaScript]
<!-- 绑定一个属性 -->
<img v-bind:src="imageSrc">

<!-- 缩写 -->
<img :src="imageSrc">

<!-- 与内联字符串连接 -->
<img :src="'/path/to/images/' + fileName ">

<!-- class绑定 -->
<div :class="{ red: isRed }"></div>
<div :class="[classA, classB]"></div>
<div :class="[classA, { classB: isB, classC: isC }]"></div>

<!-- style绑定 -->
<div :style="{ fontSize: size + 'px' }"></div>
<div :style="[styleObjectA, styleObjectB]"></div>

<!-- 绑定一个有属性的对象 -->
<div v-bind="{ id: someProp, 'other-attr': otherProp }"></div>

<!-- 通过prop修饰符绑定DOM属性 -->
<div v-bind:text-content.prop="text"></div>

<!-- prop绑定,prop必须在my-component中声明 -->
<my-component :prop="someThing"></my-component>

<!-- XLink -->
<svg><a :xlink:special="foo"></svg>
\end{lstlisting}

\begin{compactitem}
\item .camel修饰符允许在使用in-DOM模板时对v-bind属性名称进行camel化。例如,SVG的viewBox属性:

\begin{lstlisting}[language=JavaScript]
<svg :view-box.camel="viewBox"></svg>
\end{lstlisting}


\item 如果使用字符串模板或使用vue-loader/vueify进行编译,则不需要.camel。
\end{compactitem}


\section{修饰符}

在绑定prop时,prop必须在子组件中声明,v-bind支持使用不同的修饰符来指定不同的绑定类型。


\begin{compactitem}
\item .sync——双向绑定,只能用于prop绑定。
\item .once——单次绑定,只能用于prop绑定。
\item .camel——将绑定的特性名字转换回驼峰命名,只能用于普通HTML特性的绑定。
\end{compactitem}


\begin{lstlisting}[language=JavaScript]
<!-- prop绑定,prop必须在my-component组件内声明 -->
<my-component :prop="someThing"></my-component>
<!-- 双向prop绑定 -->
<my-component :prop.sync="someThing"></my-component>
<!-- 单次prop绑定 -->
<my-component :prop.once="someThing"></my-component>
\end{lstlisting}


\subsubsection{.sync}

.sync——双向绑定,只能用于prop绑定。



\subsubsection{.once}

.once——单次绑定,只能用于prop绑定。


\begin{lstlisting}[language=JavaScript]

\end{lstlisting}


\subsubsection{.camel}

.camel——将绑定的特性名字转换回驼峰命名,只能用于普通HTML特性的绑定,通常用于绑定使用驼峰命名的SVG特性(例如viewBox)。

\begin{lstlisting}[language=JavaScript]

\end{lstlisting}




\begin{lstlisting}[language=JavaScript]

\end{lstlisting}


\section{v-model}



v-model可以在表单控件(例如input、select、text、checkbox、radio和textarea)或者组件上创建双向数据绑定。


\subsection{类型}

\begin{compactitem}
\item 类型:any
\item 限制:

\begin{compactenum}
\item <input>
\item <select>
\item <textarea>
\item components
\end{compactenum}

\item 修饰符

\begin{compactenum}
\item \texttt{.lazy}:取代input监听change事件
\item \texttt{.number}:把输入的字符串转换为数字
\item \texttt{.trim}:过滤输入的首尾空格
\end{compactenum}

\end{compactitem}

v-model的实质是语法糖,在用户的输入事件中更新数据,并且特别处理一些极端情况下的数据。


\subsection{DOM}


v-model根据控件类型来自动选择正确的方法更新元素。


\begin{lstlisting}[language=JavaScript]
<html>
<head>
<script src="/path/to/vue.js"></script>
</head>
<body id="example">
  <form>
  姓名:
  <input type="text" v-model="data.name" placeholder=""><br>
  性别:
  <input type="radio" id="man" value="One" v-model="data.sex">
  <label for="man">男</label>
  <input type="radio" id="male" value="Two" v-model="data.sex">
  <label for="woman">女</label><br>
  兴趣:
  <input type="checkbox" id="book" value="book" v-model="data.interest">
  <label for="book">阅读</label>
  <input type="checkbox" id="swim" value="swim" v-model="data.interest">
  <label for="book">游泳</label>
  <input type="checkbox" id="game" value="game" v-model="data.interest">
  <label for="book">游戏</label>
  <input type="checkbox" id="song" value="sone" v-model="data.interest">
  <label for="book">唱歌</label><br>
  身份:
  <select v-model="data.identity">
    <option value="teacher" selected>教师</option>
    <option value="doctor">医生</option>
    <option value="lawyer">律师</option>
  </select>
  </form>
</body>
<script>
  new Vue({
    el: '#example',
    data: {
      data: {
        name: "",
        sex: "",
        interest: [],
        identity: ""
      }
    }
  })
</scrip>
<html>
\end{lstlisting}



\begin{lstlisting}[language=JavaScript]

\end{lstlisting}


\subsection{修饰符}

v-model支持在后面添加多个修饰符来进行定制(例如number、lazy和trim)


\subsection{.number}

v-model的number修饰符可以用户的输入自动转换为Number类型(如果原值的转换结果为NaN,则返回原值)。

\begin{lstlisting}[language=JavaScript]

\end{lstlisting}


\subsubsection{.lazy}

默认情况下,v-model在input事件中同步输入框的值和数据,lazy修饰符可以把数据修改指定到change事件中才执行。

\begin{lstlisting}[language=JavaScript]
<html>
  <head>
    <script src="/path/to/vue.js"></script>
  </head>
  <body>
  <div id="example">
    <input v-model="msg" lazy>
    <p>{{ msg }}</p>
  </div>
  </body>
  <script>
  var exampleVM = new Vue({
    el: '#example',
    data: {
      msg: 'input的内容在change事件后才改变'
    }
  })
  </script>
</html>
\end{lstlisting}

虽然输入数据时触发了input事件,lazy修饰符后也会把数据阻塞,直到触发了change事件后才修改msg的值。

\subsubsection{.trim}


\begin{lstlisting}[language=JavaScript]

\end{lstlisting}


\subsubsection{.debounce}

debounce修饰符可以设置一个最小的延时,在每次点击之后延时同步输入框的值,这样在高消耗操作(例如在input中输入内容时随时发送AJAX请求来实现自动完成)中较为有用。



\begin{lstlisting}[language=JavaScript]
<html>
  <head>
    <script src="/path/to/vue.js"></script>
  </head>
  <body>
  <div id="example">
    <input v-model="msg" debounce="5000">
    <p>{{ msg }}</p>
  </div>
  </body>
  <script>
  var exampleVM = new Vue({
    el: '#example',
    data: {
      msg: 'input的内容在5000ms后才改变'
    }
  })
  </script>
</html>
\end{lstlisting}




\begin{lstlisting}[language=JavaScript]

\end{lstlisting}




\begin{lstlisting}[language=JavaScript]

\end{lstlisting}

\section{v-pre}



v-pre不需要表达式,其用途是跳过这个元素和它的子元素的编译过程。

\subsection{类型}

v-pre可以用来显示原始 Mustache 标签,或者跳过大量没有指令的节点来加快编译。


\begin{lstlisting}[language=JavaScript]
<span v-pre>{{ this will not be compiled }}</span>
\end{lstlisting}



\begin{lstlisting}[language=JavaScript]

\end{lstlisting}




\begin{lstlisting}[language=JavaScript]

\end{lstlisting}




\begin{lstlisting}[language=JavaScript]

\end{lstlisting}




\begin{lstlisting}[language=JavaScript]

\end{lstlisting}




\begin{lstlisting}[language=JavaScript]

\end{lstlisting}




\begin{lstlisting}[language=JavaScript]

\end{lstlisting}




\begin{lstlisting}[language=JavaScript]

\end{lstlisting}

\section{v-cloak}

v-cloak不需要表达式,可以用来保持在元素上直到关联实例结束编译。和 CSS 规则(例如\texttt{[v-cloak] \{ display: none \}} 一起用时,v-pre指令可以隐藏未编译的 Mustache 标签直到实例准备完毕。

AngularJS提供了和v-cloak相同的功能。

\subsection{类型}

下面的v-cloak示例让元素不会显示,直到编译结束。

\begin{lstlisting}[language=JavaScript]
[v-cloak] {
  display: none;
}

<div v-cloak>
  {{ message }}
</div>
\end{lstlisting}



\begin{lstlisting}[language=JavaScript]

\end{lstlisting}




\begin{lstlisting}[language=JavaScript]

\end{lstlisting}




\begin{lstlisting}[language=JavaScript]

\end{lstlisting}




\begin{lstlisting}[language=JavaScript]

\end{lstlisting}




\begin{lstlisting}[language=JavaScript]

\end{lstlisting}




\begin{lstlisting}[language=JavaScript]

\end{lstlisting}




\begin{lstlisting}[language=JavaScript]

\end{lstlisting}

\section{v-once}

v-once不需要表达式,而且v-once可以只渲染元素和组件一次,这样随后的重新渲染中,元素/组件及其所有的子节点将被视为静态内容并跳过,从而可以用于优化更新性能。

\subsection{类型}


\begin{lstlisting}[language=JavaScript]
<!-- 单个元素 -->
<span v-once>This will never change: {{ msg }}</span>

<!-- 有子元素 -->
<div v-once>
  <h1>comment</h1>
  <p>{{ msg }}</p>
</div>

<!-- 组件 -->
<my-component v-once :comment="msg"></my-component>

<!-- v-for指令 -->
<ul>
  <li v-for="i in list" v-once>{{ i }}</li>
</ul>
\end{lstlisting}

\section{directive}


Vue.js允许在内置指令之外注册自定义指令,自定义指令提供一种机制将数据的变化映射为DOM行为。


\begin{compactitem}
\item AngularJS的指令是使用\texttt{Pdirective(name,factory\_function)}实现的。



\begin{lstlisting}[language=JavaScript]
angular.module('myapp',[],)
.directive(myDirective,function(){
  return {
     template: '',
     restrict:'',
     template:'',
     replace:'',
     //...
  }
})
\end{lstlisting}

\item Vue.js的指令是使用\texttt{Vue.directive(id,definition)}方法注册一个全局自定义指令。

\end{compactitem}

Vue.directive(id,definition)接受两个参数:指令ID和定义对象,也可以使用组件的directives选项注册一个局部自定义指令(相当于AngularJS restrict属性值为A)。


\subsection{Hook}

AngularJS提供了两个函数——compile和link,其中:

\begin{compactitem}
\item compile函数主要负责把作用域和DOM进行链接;
\item link函数用来创建可以操作DOM的指令。
\end{compactitem}

compile和link的选项是互斥的,如果同时设置这两个选项,则会把compile返回的函数当作link函数,而忽略link选项本身。



Vue.js提供的对应的函数都是可选的,相互之间没有制约关系。

\begin{compactitem}
\item bind——只调用一次,在指令第一次绑定到元素上时调用。


\item update——在bind之后立即以初始值为参数第一次调用,之后每当绑定值变化时都会调用,参数为新值和旧值。


\item unbind——只调用一次,在指令从元素上解绑时调用。

\end{compactitem}



\begin{lstlisting}[language=JavaScript]
Vue.directive('my-directive',{
   bind: function () {
      // 准备工作,例如添加事件处理器或只需要运行一次的高耗工作
   },
   
   update: function (newValue, oldValue) {
     // 值更新时的工作,也会以初始值为参数调用一次
   },
   
   unbind: function () {
     // 清理工作,例如删除bind()添加的事件监听器
   }
})
\end{lstlisting}


在注册指令后就可以在Vue.js的模板中这样使用,需要添加前缀\texttt{v-},例如:


\begin{lstlisting}[language=JavaScript]
<div v-my-directive="someValue"></div>
\end{lstlisting}

如果只需要update函数,可以传入一个函数替代自定义对象:


\begin{lstlisting}[language=JavaScript]
Vue.directive('my-directive', function(value){
  // 这个函数用作update()
})
\end{lstlisting}

\subsubsection{bind}





\subsubsection{update}


\subsubsection{unbind}



\begin{lstlisting}[language=JavaScript]

\end{lstlisting}




\begin{lstlisting}[language=JavaScript]

\end{lstlisting}




\begin{lstlisting}[language=JavaScript]

\end{lstlisting}




\begin{lstlisting}[language=JavaScript]

\end{lstlisting}




\begin{lstlisting}[language=JavaScript]

\end{lstlisting}