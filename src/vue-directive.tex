\part{Vue Directive}


\chapter{Overview}

\begin{compactitem}
\item 类型:
\end{compactitem}

\begin{lstlisting}[language=JavaScript]

\end{lstlisting}



\begin{lstlisting}[language=JavaScript]

\end{lstlisting}




\begin{lstlisting}[language=JavaScript]

\end{lstlisting}




\begin{lstlisting}[language=JavaScript]

\end{lstlisting}




\begin{lstlisting}[language=JavaScript]

\end{lstlisting}




\begin{lstlisting}[language=JavaScript]

\end{lstlisting}




\begin{lstlisting}[language=JavaScript]

\end{lstlisting}




\begin{lstlisting}[language=JavaScript]

\end{lstlisting}

\chapter{v-text}


v-text可以更新元素的 textContent。如果要更新部分的 textContent ,需要使用 \{\{ Mustache \}\} 插值。

\begin{compactitem}
\item 类型:string
\end{compactitem}


\begin{lstlisting}[language=JavaScript]
<span v-text="msg"></span>
<!-- 和下面一样 -->
<span>{{ msg }}</span>
\end{lstlisting}



\begin{lstlisting}[language=JavaScript]

\end{lstlisting}




\begin{lstlisting}[language=JavaScript]

\end{lstlisting}




\begin{lstlisting}[language=JavaScript]

\end{lstlisting}




\begin{lstlisting}[language=JavaScript]

\end{lstlisting}




\begin{lstlisting}[language=JavaScript]

\end{lstlisting}




\begin{lstlisting}[language=JavaScript]

\end{lstlisting}




\begin{lstlisting}[language=JavaScript]

\end{lstlisting}

\chapter{v-html}

v-html可以更新元素的 innerHTML 。

注意,内容按普通 HTML 插入,不会作为 Vue 模板进行编译 。

如果试图使用 v-html 组合模板,可以重新考虑是否通过使用组件(Component)来替代。

\begin{compactitem}
\item 类型:string
\end{compactitem}

在网站上动态渲染任意 HTML 是非常危险的,因为容易导致 XSS 攻击。

仅仅只在可信内容上使用 v-html,永远不要用在用户提交的内容上。


\begin{lstlisting}[language=JavaScript]
<div v-html="html"></div>
<!-- 和下面的相同 -->
<div>{{{ html }}}</div>
\end{lstlisting}



\begin{lstlisting}[language=JavaScript]

\end{lstlisting}




\begin{lstlisting}[language=JavaScript]

\end{lstlisting}




\begin{lstlisting}[language=JavaScript]

\end{lstlisting}




\begin{lstlisting}[language=JavaScript]

\end{lstlisting}




\begin{lstlisting}[language=JavaScript]

\end{lstlisting}




\begin{lstlisting}[language=JavaScript]

\end{lstlisting}




\begin{lstlisting}[language=JavaScript]

\end{lstlisting}

\chapter{v-show}

v-show可以根据表达式的真假值来切换元素的display CSS 属性。

\begin{compactitem}
\item 类型:any
\end{compactitem}

当条件变化时该指令触发过渡效果。


\begin{lstlisting}[language=JavaScript]

\end{lstlisting}



\begin{lstlisting}[language=JavaScript]

\end{lstlisting}




\begin{lstlisting}[language=JavaScript]

\end{lstlisting}




\begin{lstlisting}[language=JavaScript]

\end{lstlisting}




\begin{lstlisting}[language=JavaScript]

\end{lstlisting}




\begin{lstlisting}[language=JavaScript]

\end{lstlisting}




\begin{lstlisting}[language=JavaScript]

\end{lstlisting}




\begin{lstlisting}[language=JavaScript]

\end{lstlisting}

\chapter{v-if}

v-if可以根据表达式的值的真假条件渲染元素。

在v-if切换时元素及它的数据绑定/组件被销毁并重建。如果元素是 <template> ,将提出它的内容作为条件块。

\begin{compactitem}
\item 类型:any
\end{compactitem}

当条件变化时该指令触发过渡效果。



\begin{lstlisting}[language=JavaScript]

\end{lstlisting}



\begin{lstlisting}[language=JavaScript]

\end{lstlisting}




\begin{lstlisting}[language=JavaScript]

\end{lstlisting}




\begin{lstlisting}[language=JavaScript]

\end{lstlisting}




\begin{lstlisting}[language=JavaScript]

\end{lstlisting}




\begin{lstlisting}[language=JavaScript]

\end{lstlisting}




\begin{lstlisting}[language=JavaScript]

\end{lstlisting}




\begin{lstlisting}[language=JavaScript]

\end{lstlisting}

\chapter{v-else}

v-else可以为 v-if 或者 v-else-if 添加 “else 块”。



\begin{compactitem}
\item 类型:any
\end{compactitem}


\begin{lstlisting}[language=JavaScript]
<div v-if="Math.random() > 0.5">
  Now you see me
</div>
<div v-else>
  Now you don't
</div>
\end{lstlisting}

\begin{compactitem}
\item 不需要表达式
\item 前一兄弟元素必须有 v-if 或 v-else-if
\end{compactitem}

\begin{lstlisting}[language=JavaScript]

\end{lstlisting}




\begin{lstlisting}[language=JavaScript]

\end{lstlisting}




\begin{lstlisting}[language=JavaScript]

\end{lstlisting}




\begin{lstlisting}[language=JavaScript]

\end{lstlisting}




\begin{lstlisting}[language=JavaScript]

\end{lstlisting}




\begin{lstlisting}[language=JavaScript]

\end{lstlisting}




\begin{lstlisting}[language=JavaScript]

\end{lstlisting}


\chapter{v-else-if}


v-else-if可以为v-if表示“else if块”,支持链式使用。


\begin{compactitem}
\item 类型:any
\end{compactitem}





\begin{lstlisting}[language=JavaScript]
<div v-if="type === 'A'">
   A
</div>
<div v-else-if="type ==='B'">
   B
</div>
<div v-else-if="type === 'C'">
   C
</div>
<div v-else>
   Not A/B/C
</div>
\end{lstlisting}

\begin{compactitem}
\item 前一个同级元素必须具有v-if或v-else-if。
\end{compactitem}

\begin{lstlisting}[language=JavaScript]

\end{lstlisting}




\begin{lstlisting}[language=JavaScript]

\end{lstlisting}




\begin{lstlisting}[language=JavaScript]

\end{lstlisting}




\begin{lstlisting}[language=JavaScript]

\end{lstlisting}




\begin{lstlisting}[language=JavaScript]

\end{lstlisting}




\begin{lstlisting}[language=JavaScript]

\end{lstlisting}




\begin{lstlisting}[language=JavaScript]

\end{lstlisting}

\chapter{v-for}

v-for可以基于源数据多次渲染元素或模板块。

\begin{compactitem}
\item 类型:Array|Object|number|string
\end{compactitem}

v-for指令的值必须使用特定语法\texttt{alias in expression}来为当前遍历的元素提供别名:


\begin{lstlisting}[language=JavaScript]
<div v-for="item in items">
  {{ item.text }}
</div>
\end{lstlisting}

另外,也可以为数组索引指定别名(或者用于对象的键):



\begin{lstlisting}[language=JavaScript]
<div v-for="(item, index) in items"></div>
<div v-for="(val,key) in object"></div>
<div v-for="(val,key,index) in object"></div>
\end{lstlisting}

v-fom默认行为将尝试不改变整体,只是替换元素。如果需要让v-for重新排序的元素,需要提供一个key的特殊属性,例如:


\begin{lstlisting}[language=JavaScript]
<div v-for="item in items" :key="item.id">
  {{ item.text }}
</div>
\end{lstlisting}




\begin{lstlisting}[language=JavaScript]

\end{lstlisting}




\begin{lstlisting}[language=JavaScript]

\end{lstlisting}




\begin{lstlisting}[language=JavaScript]

\end{lstlisting}




\begin{lstlisting}[language=JavaScript]

\end{lstlisting}




\begin{lstlisting}[language=JavaScript]

\end{lstlisting}

\chapter{v-on}

v-on可以绑定事件监听器,v-on绑定的事件类型由参数指定。


\begin{lstlisting}[language=JavaScript]
<!-- 方法处理器 -->
<button v-on:click="doThis"></button>
<!-- 缩写 -->
<button @click="doThis"></button>
\end{lstlisting}


\begin{compactitem}
\item 缩写:@
\item 类型:Function|Inline Statement
\item 参数:event(必选)
\end{compactitem}


如果使用内联语句,语句可以访问一个 \$event 属性。例如,\texttt{v-on:click="handle('ok', \$event)"}。

\begin{lstlisting}[language=JavaScript]
<!-- 内联语句 -->
<button v-on:click="doThat('hello',$event)"></button>
\end{lstlisting}

\begin{compactitem}
\item v-on用在普通元素上时,只能监听 原生 DOM 事件。

在监听原生 DOM 事件时,方法以事件为唯一的参数。

\item v-on用在自定义元素组件上时,也可以监听子组件触发的自定义事件。
\end{compactitem}

v-on支持的修饰符如下:


\begin{compactitem}
\item .stop:调用event.stopPropagation()
\item .prevent:调用event.preventDefault()
\item .capture:添加事件监听器时使用capture模式
\item .self:只当事件是从监听器绑定的元素本身触发时才触发回调
\item .\{keyCode|keyAlias\}:只当事件是从监听器绑定的元素本身触发时才触发回调
\item .native:监听组件根元素的原生事件
\end{compactitem}

表达式可以是一个方法的名字或一个内联语句,如果没有修饰符也可以省略。



\begin{lstlisting}[language=JavaScript]
<!-- 方法处理器 -->
<button v-on:click="doThis"></button>

<!-- 内联语句 -->
<button v-on:click="doThat('hello',$event)"></button>

<!-- 缩写 -->
<button @click="doThis"></button>

<!-- 停止冒泡 -->
<button @click.stop="doThis"></button>

<!-- 阻止默认行为 -->
<button @click.prevent="doThis"></button>

<!-- 阻止默认行为,没有表达式 -->
<form @submit.prevent></form>

<!-- 串联修饰符 -->
<button @click.stop.prevent="doThis"></button>

<!-- 键修饰符,键别名 -->
<input @keyup.enter="onEnter">

<!-- 键修饰符,键代码 -->
<input @keyup.13="onEnter''>
\end{lstlisting}

在子组件上监听自定义事件(当子组件触发 “my-event” 时将调用事件处理器):

\begin{lstlisting}[language=JavaScript]
<my-component @my-event="handleThis''></my-component>

<!-- 内联语句 -->
<my-component @my-event="handleThis(123,$event)"></my-component>

<!-- 组件中的原生事件 -->
<my-component @click.native="onClick"></my-component>
\end{lstlisting}




\begin{lstlisting}[language=JavaScript]

\end{lstlisting}




\begin{lstlisting}[language=JavaScript]

\end{lstlisting}




\begin{lstlisting}[language=JavaScript]

\end{lstlisting}




\begin{lstlisting}[language=JavaScript]

\end{lstlisting}




\begin{lstlisting}[language=JavaScript]

\end{lstlisting}




\begin{lstlisting}[language=JavaScript]

\end{lstlisting}

\chapter{v-bind}

v-bind可以动态地绑定一个或多个特性,或一个组件 prop 到表达式。

\begin{compactitem}
\item 缩写:\texttt{:}
\item 类型:any(with argument)|Object(without argument)
\item 参数:attrOrProp(可选)
\end{compactitem}



\begin{lstlisting}[language=JavaScript]
<!-- 绑定一个属性 -->
<img v-bind:src="imageSrc">

<!-- 缩写 -->
<img :src="imageSrc">
\end{lstlisting}


\begin{compactitem}
\item v-bind在绑定 class 或 style 特性时,支持其它类型的值(例如数组或对象)。


\begin{lstlisting}[language=JavaScript]
<!-- class绑定 -->
<div :class="{ red: isRed }"></div>
<div :class="[classA, classB]"></div>
<div :class="[classA, { classB: isB, classC: isC }]"></div>

<!-- style绑定 -->
<div :style="{ fontSize: size + 'px' }"></div>
<div :style="[styleObjectA, styleObjectB]"></div>
\end{lstlisting}

\item v-bind在绑定 prop 时,prop 必须在子组件中声明。可以用修饰符指定不同的绑定类型。

\begin{lstlisting}[language=JavaScript]
<!-- 绑定一个有属性的对象 -->
<div v-bind="{ id: someProp, 'other-attr': otherProp }"></div>

<!-- 通过prop修饰符绑定DOM属性 -->
<div v-bind:text-content.prop="text"></div>

<!-- prop绑定,prop必须在my-component中声明 -->
<my-component :prop="someThing"></my-component>
\end{lstlisting}

\end{compactitem}



\begin{lstlisting}[language=JavaScript]
<!-- 绑定一个属性 -->
<img v-bind:src="imageSrc">

<!-- 缩写 -->
<img :src="imageSrc">

<!-- 与内联字符串连接 -->
<img :src="'/path/to/images/' + fileName ">

<!-- class绑定 -->
<div :class="{ red: isRed }"></div>
<div :class="[classA, classB]"></div>
<div :class="[classA, { classB: isB, classC: isC }]"></div>

<!-- style绑定 -->
<div :style="{ fontSize: size + 'px' }"></div>
<div :style="[styleObjectA, styleObjectB]"></div>

<!-- 绑定一个有属性的对象 -->
<div v-bind="{ id: someProp, 'other-attr': otherProp }"></div>

<!-- 通过prop修饰符绑定DOM属性 -->
<div v-bind:text-content.prop="text"></div>

<!-- prop绑定,prop必须在my-component中声明 -->
<my-component :prop="someThing"></my-component>

<!-- XLink -->
<svg><a :xlink:special="foo"></svg>
\end{lstlisting}

\begin{compactitem}
\item .camel修饰符允许在使用in-DOM模板时对v-bind属性名称进行camel化。例如,SVG的viewBox属性:

\begin{lstlisting}[language=JavaScript]
<svg :view-box.camel="viewBox"></svg>
\end{lstlisting}


\item 如果使用字符串模板或使用vue-loader/vueify进行编译,则不需要.camel。
\end{compactitem}



\begin{lstlisting}[language=JavaScript]

\end{lstlisting}




\begin{lstlisting}[language=JavaScript]

\end{lstlisting}




\begin{lstlisting}[language=JavaScript]

\end{lstlisting}




\begin{lstlisting}[language=JavaScript]

\end{lstlisting}


\chapter{v-model}

v-model可以在表单控件(例如input、select和textarea)或者组件上创建双向绑定。


\begin{compactitem}
\item 类型:any
\item 限制:

\begin{compactenum}
\item <input>
\item <select>
\item <textarea>
\item components
\end{compactenum}

\item 修饰符

\begin{compactenum}
\item \texttt{.lazy}:取代input监听change事件
\item \texttt{.number}:把输入的字符串转换为数字
\item \texttt{.trim}:过滤输入的首尾空格
\end{compactenum}

\end{compactitem}



\begin{lstlisting}[language=JavaScript]

\end{lstlisting}



\begin{lstlisting}[language=JavaScript]

\end{lstlisting}




\begin{lstlisting}[language=JavaScript]

\end{lstlisting}




\begin{lstlisting}[language=JavaScript]

\end{lstlisting}




\begin{lstlisting}[language=JavaScript]

\end{lstlisting}




\begin{lstlisting}[language=JavaScript]

\end{lstlisting}




\begin{lstlisting}[language=JavaScript]

\end{lstlisting}




\begin{lstlisting}[language=JavaScript]

\end{lstlisting}

\chapter{v-pre}



v-pre不需要表达式,其用途是跳过这个元素和它的子元素的编译过程。

v-pre可以用来显示原始 Mustache 标签,或者跳过大量没有指令的节点来加快编译。


\begin{lstlisting}[language=JavaScript]
<span v-pre>{{ this will not be compiled }}</span>
\end{lstlisting}



\begin{lstlisting}[language=JavaScript]

\end{lstlisting}




\begin{lstlisting}[language=JavaScript]

\end{lstlisting}




\begin{lstlisting}[language=JavaScript]

\end{lstlisting}




\begin{lstlisting}[language=JavaScript]

\end{lstlisting}




\begin{lstlisting}[language=JavaScript]

\end{lstlisting}




\begin{lstlisting}[language=JavaScript]

\end{lstlisting}




\begin{lstlisting}[language=JavaScript]

\end{lstlisting}

\chapter{v-cloak}

v-cloak不需要表达式,可以用来保持在元素上直到关联实例结束编译。和 CSS 规则(例如\texttt{[v-cloak] \{ display: none \}} 一起用时,v-pre指令可以隐藏未编译的 Mustache 标签直到实例准备完毕。

下面的v-cloak示例让元素不会显示,直到编译结束。

\begin{lstlisting}[language=JavaScript]
[v-cloak] {
  display: none;
}

<div v-cloak>
  {{ message }}
</div>
\end{lstlisting}



\begin{lstlisting}[language=JavaScript]

\end{lstlisting}




\begin{lstlisting}[language=JavaScript]

\end{lstlisting}




\begin{lstlisting}[language=JavaScript]

\end{lstlisting}




\begin{lstlisting}[language=JavaScript]

\end{lstlisting}




\begin{lstlisting}[language=JavaScript]

\end{lstlisting}




\begin{lstlisting}[language=JavaScript]

\end{lstlisting}




\begin{lstlisting}[language=JavaScript]

\end{lstlisting}

\chapter{v-once}

v-once不需要表达式,而且v-once可以只渲染元素和组件一次,这样随后的重新渲染中,元素/组件及其所有的子节点将被视为静态内容并跳过,从而可以用于优化更新性能。


\begin{lstlisting}[language=JavaScript]
<!-- 单个元素 -->
<span v-once>This will never change: {{ msg }}</span>

<!-- 有子元素 -->
<div v-once>
  <h1>comment</h1>
  <p>{{ msg }}</p>
</div>

<!-- 组件 -->
<my-component v-once :comment="msg"></my-component>

<!-- v-for指令 -->
<ul>
  <li v-for="i in list" v-once>{{ i }}</li>
</ul>
\end{lstlisting}



\begin{lstlisting}[language=JavaScript]

\end{lstlisting}




\begin{lstlisting}[language=JavaScript]

\end{lstlisting}




\begin{lstlisting}[language=JavaScript]

\end{lstlisting}




\begin{lstlisting}[language=JavaScript]

\end{lstlisting}




\begin{lstlisting}[language=JavaScript]

\end{lstlisting}




\begin{lstlisting}[language=JavaScript]

\end{lstlisting}




\begin{lstlisting}[language=JavaScript]

\end{lstlisting}