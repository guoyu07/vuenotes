\part{Vue Binding}


\chapter{Overview}

数据绑定就是把数据和视图进行关联,当数据发生变化时可以自动更新视图。Vue.js把数据绑定的语法设计为可配置的,可以在Vue.config全局配置上修改数据绑定的语法。



\begin{lstlisting}[language=JavaScript]
let delimiters = ['{{','}}']
let unsafeDelimiters = ['{{{','}}}']
\end{lstlisting}

Vue.config是一个包含了Vue.js所有的全局配置的对象,可以在Vue.js实例化之前修改其中的属性。



\begin{lstlisting}[language=JavaScript]
Vue.config.delimeters = ['<%','%>]
Vue.config.unsafeDelimeters = ['<$','$>']
\end{lstlisting}

不同的JavaScript框架都提供了自己的数据绑定语法,Vue.js提供了不同的文本渲染方式来满足日常的模板渲染需求。




数据绑定一个常见需求是操作元素的 class 列表和它的内联样式,class和style都是属性 ,可以用v-bind 处理它们——只需要计算出表达式最终的字符串。

不过,字符串拼接麻烦又易错,在 v-bind 用于 class 和 style 时, Vue.js 专门增强了它,现在表达式的结果类型除了字符串之外,还可以是对象或数组。

\section{Mustache}

Vue.js支持使用Mustache标签(也就是双大括号\{\{\}\})来来处理文本插值。


\begin{lstlisting}[language=JavaScript]
<span>Text: {{ text }}</span>
\end{lstlisting}

上述的标签\{\{text\}\}被相应的数据对象的text属性的值替换,而且当text的值改变时,文本中的值也会联动地发生变化。

如果数据只需要渲染一次,后续不再更新,可以使用\texttt{"*"}来进行指定:

\begin{lstlisting}[language=JavaScript]
<span>{{ *text }}</span>
\end{lstlisting}

默认情况下,\{\{\}\}会把内部的值全部作为字符串进行处理,如果值是HTML片段,可以使用三个大括号来绑定,例如:


\begin{lstlisting}[language=JavaScript]
<div>Logo: {{{ logo }}}</div>

logo: '<span>Hello</span>'
\end{lstlisting}

Vue.js支持把\{\{\}\}放在HTML标签内,例如:

\begin{lstlisting}[language=JavaScript]
<li data-id='{{ id }}'></li>
\end{lstlisting}


注意,Vue.js指令和自身特性内是不可以插值的,否则Vue.js会发出警告。


文本插值支持字符串、HTML片段以及表达式形式的值,表达式可以由JavaScript表达式和过滤器构成,而且过滤器可以没有,也可以有多个。

复杂的JavaScript表达式可以是数值、变量和运算符的综合体,简单的JavaScript表达式可以是常量或变量名称,表达式的值是其运算结果。



\begin{lstlisting}[language=JavaScript]
{{ cents/10 }}
{{ true ? 1 : 0 }}
{{ example.split(' ') }}
\end{lstlisting}

下面的插值是不合法的:

\begin{lstlisting}[language=JavaScript]
{{ var logo = 'hello' }} // 语句不是表达式
{{ if(true) return 'world' }} // 只支持三元式,不支持条件控制语句
\end{lstlisting}

Vue.js支持在表达式后面添加过滤符来实现类似Linux中的管道的功能的过滤器,例如:


\begin{lstlisting}[language=JavaScript]
{{ example | toUpperCase }}
\end{lstlisting}

JavaScript过滤器本质上是一个JavaScript函数,而且Vue.js允许串联过滤器。

\begin{lstlisting}[language=JavaScript]
{{ example | filterA | filterB }}
\end{lstlisting}

Vue.js支持带有参数的过滤器,参数之间使用空格隔开,例如:


\begin{lstlisting}[language=JavaScript]
{{ example | filter a b }}
\end{lstlisting}

Vue.js提供了一些内置的过滤器。

\section{Directive}


Vue.js提供的指令可以理解为带有v-前缀的特殊特性,指令的值限定为绑定表达式,即JavaScript表达式和过滤器。

Vue.js指令的职责就是当表达式的值发生变化时,联动地把某些特殊的行为反映到DOM上来实现视图更新,例如:


\begin{lstlisting}[language=JavaScript]
<div v-if="show">Hello</div>
\end{lstlisting}


这里,如果show为true则显示Hello,否则就不显示。

Vue.js还支持在指令和表达式之间插入一个用冒号分隔的参数,例如:

\begin{lstlisting}[language=JavaScript]
<a v-bind:href="url"></a>
<div v-on:click="action"></div>
\end{lstlisting}

Vue.js提供了一些内置的指令,例如:

\begin{compactitem}
\item v-el
\item v-on
\item v-bind
\item v-cloak
\item v-if
\item v-else
\item v-show
\item v-for
\item v-repeat
\item v-model
\item v-text
\item v-html
\item v-ref
\item v-pre
\end{compactitem}

\subsection{v-if}


\section{Binding HTML Classes}



\subsection{Object Syntax}

可以传给\texttt{v-bind:class} 一个对象来动态地切换 class 。

\begin{lstlisting}[language=JavaScript]
<div v-bind:class="{ active: isActive }"></div>
\end{lstlisting}

上面的语法表示 class~active 的更新将取决于数据属性 isActive 是否为真值 。

也可以在对象中传入更多属性用来动态切换多个 class 。

此外, \texttt{v-bind:class}指令可以与普通的 class 属性共存。

\begin{compactitem}
\item 模板

\begin{lstlisting}[language=JavaScript]
<div class="static"
     v-bind:class="{ active: isActive, 'text-danger': hasError }">
</div>
\end{lstlisting}

\item 数据

\begin{lstlisting}[language=JavaScript]
data: {
  isActive: true,
  hasError: false
}
\end{lstlisting}
\end{compactitem}



最终被渲染为:


\begin{lstlisting}[language=JavaScript]
<div class="static active"></div>
\end{lstlisting}

当 isActive 或者 hasError 变化时,class 列表将相应地更新。例如,如果 hasError 的值为 true , class列表将变为\texttt{"static active text-danger"}。

也可以直接绑定数据里的一个对象:

\begin{lstlisting}[language=JavaScript]
<div v-bind:class="classObject"></div>
\end{lstlisting}

对应的数据为:

\begin{lstlisting}[language=JavaScript]
data: {
  classObject: {
    active: true,
    'text-danger': false
  }
}
\end{lstlisting}

最终的渲染的结果和上面一样,也可以在这里绑定返回对象的计算属性,这是一个常用且强大的模式:

\begin{lstlisting}[language=JavaScript]
<div v-bind:class="classObject"></div>

<script>
data: {
  isActive: true,
  error: null
},
computed: {
  classObject: function () {
    return {
      active: this.isActive && !this.error,
      'text-danger': this.error && this.error.type === 'fatal',
    }
  }
}
</script>
\end{lstlisting}


\subsection{Array Syntax}

可以把一个数组传给 \texttt{v-bind:class} 以应用一个 class 列表:


\begin{lstlisting}[language=JavaScript]
<div v-bind:class="[activeClass, errorClass]">
\end{lstlisting}

对应的数据:

\begin{lstlisting}[language=JavaScript]
data: {
  activeClass: 'active',
  errorClass: 'text-danger'
}
\end{lstlisting}

最终被渲染为:

\begin{lstlisting}[language=JavaScript]
<div class="active text-danger"></div>
\end{lstlisting}

如果需要根据条件切换列表中的 class ,可以用三元表达式:


\begin{lstlisting}[language=JavaScript]
<div v-bind:class="[isActive ? activeClass : '', errorClass]">
\end{lstlisting}

此例始终添加 errorClass ,但是只有在 isActive 是 true 时添加 activeClass 。

不过,当有多个条件 class 时这样写有些繁琐。可以在数组语法中使用对象语法:

\begin{lstlisting}[language=JavaScript]
<div v-bind:class="[{ active: isActive }, errorClass]">
\end{lstlisting}




\subsection{With Components}

在一个定制的组件上用到 class 属性的时候,这些类将被添加到根元素上面,这个元素上已经存在的类不会被覆盖。

例如,在声明一个组件my-component来使用class属性:

\begin{lstlisting}[language=JavaScript]
Vue.component('my-component',{
   template: '<p class="foo bar">Hi</p>'
})
\end{lstlisting}

在使用它的时候添加一些 class:

\begin{lstlisting}[language=JavaScript]
<my-component class="baz boo"></my-component>
\end{lstlisting}

HTML 最终将被渲染成为:


\begin{lstlisting}[language=JavaScript]
<p class="foo bar baz boo">Hi</p>
\end{lstlisting}

同样的适用于绑定 HTML class :

\begin{lstlisting}[language=JavaScript]
<my-component v-bind:class="{ active: isActive }"></my-component>
\end{lstlisting}

当 isActive 为 true 的时候,HTML 将被渲染成为:


\begin{lstlisting}[language=JavaScript]
<p class="foo bar active"></p>
\end{lstlisting}








\section{Binding Inline Styles}



\subsection{Object Syntax}

v-bind:style 的对象语法十分直观——看着非常像 CSS ,其实它是一个 JavaScript 对象。 CSS 属性名可以用驼峰式(camelCase)或短横分隔命名(kebab-case):


\begin{lstlisting}[language=JavaScript]
<div v-bind:style="{ color: activeColor, fontSize: fontSize + 'px' }"></div>

<script>
data: {
  activeColor: 'red',
  fontSize: 30
}
</script>
\end{lstlisting}

直接绑定到一个样式对象通常更好,让模板更清晰:


\begin{lstlisting}[language=JavaScript]
<div v-bind:style="styleObject"></div>

<script>
data: {
  styleObject: {
    color: 'red',
    fontSize: '13px'
  }
}
</script>
\end{lstlisting}

同样的,对象语法常常结合返回对象的计算属性使用。

\begin{lstlisting}[language=JavaScript]

\end{lstlisting}



\subsection{Array Syntax}

v-bind:style 的数组语法可以将多个样式对象应用到一个元素上:


\begin{lstlisting}[language=JavaScript]
<div v-bind:style="[baseStyles, overridingStyles]">
\end{lstlisting}





\subsection{Auto Prefixing}

当 v-bind:style 使用需要特定前缀的 CSS 属性时,如 transform ,Vue.js 会自动侦测并添加相应的前缀。

