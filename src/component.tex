\part{Vue Component}


\chapter{Overview}




Vue全局组件需要使用 \texttt{Vue.component}来定义,并且紧接着用 \texttt{new Vue(\{ el: '\#container '\})} 在每个页面内指定一个容器元素。



\begin{lstlisting}[language=JavaScript]
<div id="example">
  <my-component></my-component>
</div>

<script>
// 全局注册
Vue.component('my-component',{
  template: '<div>A custom component!</div>'
});

// 创建根实例
new Vue({
    el: '#example'
})
</script>
\end{lstlisting}

\begin{compactitem}
\item 全局定义强制要求每个 component 中的命名不得重复;
\item 字符串模板无法语法高亮,在 HTML 有多行时需要用到丑陋的 \textbackslash;
\item 不支持CSS意味着当 HTML 和 JavaScript 组件化时,CSS 明显被遗漏
\item 没有构建步骤限制只能使用 HTML 和 ES5 JavaScript,不能使用预处理器(例如Pug和Babel)
\end{compactitem}



Vue.js提供的单文件组件(文件扩展名为 .vue)为上述所有问题提供了解决方案,Vue.js 本身提供了配套工具(例如vue-cli)来开发单文件组件。

具体来说,每个.vue文件就是一个组件,同时还包含了组件之间的依赖关系,这样可以使用 Webpack 或 Browserify 等构建工具。

Vue单文件组件集成了组件需要的样式(style)、模板(template)和脚本(script),支持完整的语法高亮、CommonJS模块以及组件化的CSS。例如,下面是一个文件名为 Hello.vue 的简单的单文件Vue组件实例:

\begin{lstlisting}[language=JavaScript]
<template>
  <p>{{ greeting }} World!</p>
</template>

<script>
module.exports = {
   data: function () {
       return {
           greeting: 'Hello'
       }
   }
}
</script>

<style scoped>
p {
   font-size: 2em;
   text-align: center;
}
</style>
\end{lstlisting}




Vue的单文件组件支持使用预处理器来构建简洁和功能更丰富的组件,比如 Jade,Babel (with ES2015 modules)和 Stylus,而且实际上在单文件组件中还可以简单地使用 Buble,TypeScript,SCSS,PostCSS 或者其他任何能够帮助提高生产力的预处理器。

Vue单文件组件的组合可以实现一个完整的使用 .vue 组件、ES2015 和 热重载( hot-reloading ) 的Vue项目。例如,在Vue.js模板中结合Webpack可以加载多个模块,并且将多个模块打包成最终Vue应用。



在 Webpack中,每个模块在被打包到 bundle 之前都由一个相应的 “loader” 来转换,Vue 也提供了 vue-loader 插件来执行 .vue 单文件组件的转换。

\begin{compactitem}
\item \texttt{npm run dev}使用Webpack+vue-loader实现代码映射和热重载;
\item \texttt{npm run build}最小化构建HTML/CSS/Javascript。
\end{compactitem}



\section{Template}



\begin{lstlisting}[language=JavaScript]
<template>
  <p>{{ greeting }} World!</p>
</template>
\end{lstlisting}


\section{Script}


\begin{lstlisting}[language=JavaScript]
<script>
module.exports = {
   data: function () {
       return {
           greeting: 'Hello'
       }
   }
}
</script>
\end{lstlisting}

\section{Style}



\begin{lstlisting}[language=JavaScript]
<style scoped>
p {
   font-size: 2em;
   text-align: center;
}
</style>
\end{lstlisting}


\chapter{Hikitchen}


\begin{lstlisting}[language=JavaScript]

\end{lstlisting}




\begin{lstlisting}[language=JavaScript]

\end{lstlisting}






\begin{lstlisting}[language=JavaScript]

\end{lstlisting}






\begin{lstlisting}[language=JavaScript]

\end{lstlisting}



\begin{lstlisting}[language=JavaScript]

\end{lstlisting}




\begin{lstlisting}[language=JavaScript]

\end{lstlisting}




\begin{lstlisting}[language=JavaScript]

\end{lstlisting}




\begin{lstlisting}[language=JavaScript]

\end{lstlisting}





\begin{lstlisting}[language=JavaScript]

\end{lstlisting}




\begin{lstlisting}[language=JavaScript]

\end{lstlisting}






\begin{lstlisting}[language=JavaScript]

\end{lstlisting}







\begin{lstlisting}[language=JavaScript]

\end{lstlisting}



\begin{lstlisting}[language=JavaScript]

\end{lstlisting}




\begin{lstlisting}[language=JavaScript]

\end{lstlisting}




\begin{lstlisting}[language=JavaScript]

\end{lstlisting}




\begin{lstlisting}[language=JavaScript]

\end{lstlisting}





\begin{lstlisting}[language=JavaScript]

\end{lstlisting}




\begin{lstlisting}[language=JavaScript]

\end{lstlisting}






\begin{lstlisting}[language=JavaScript]

\end{lstlisting}





\begin{lstlisting}[language=JavaScript]

\end{lstlisting}



\begin{lstlisting}[language=JavaScript]

\end{lstlisting}




\begin{lstlisting}[language=JavaScript]

\end{lstlisting}




\begin{lstlisting}[language=JavaScript]

\end{lstlisting}




\begin{lstlisting}[language=JavaScript]

\end{lstlisting}





\begin{lstlisting}[language=JavaScript]

\end{lstlisting}




\begin{lstlisting}[language=JavaScript]

\end{lstlisting}






\begin{lstlisting}[language=JavaScript]

\end{lstlisting}