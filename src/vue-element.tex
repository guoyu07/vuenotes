\part{Vue Element}


\chapter{Overview}


\chapter{key}

key(string) 的特殊属性主要用在 Vue的虚拟DOM算法,在新旧nodes对比时辨识VNodes。

\begin{compactitem}
\item 如果不使用key,Vue会使用一种最大限度减少动态元素并且尽可能的尝试修复/再利用相同类型元素的算法。
\item 使用key,Vue会基于key的变化重新排列元素顺序,并且会移除key不存在的元素。
\end{compactitem}



有相同父元素的子元素必须有唯一的key,否则重复的key会造成渲染错误。

key的最常见的用例是结合 v-for:



\begin{lstlisting}[language=JavaScript]
<ul>
  <li v-for="item in items" :key="item.id"></li>
</ul>
\end{lstlisting}

key也可以用于强制替换元素/组件而不是重复使用它,当遇到如下场景时它可能会很有用:

\begin{compactitem}
\item 完整地触发组件的生命周期钩子
\item 触发过渡
\end{compactitem}

下面的示例中,当text发生改变时,<span>会随时被更新,因此会触发过渡。

\begin{lstlisting}[language=JavaScript]
<transition>
  <span :key="text">{{ text }}</span>
</transition>
\end{lstlisting}




\begin{lstlisting}[language=JavaScript]

\end{lstlisting}




\begin{lstlisting}[language=JavaScript]

\end{lstlisting}




\begin{lstlisting}[language=JavaScript]

\end{lstlisting}




\begin{lstlisting}[language=JavaScript]

\end{lstlisting}



\chapter{ref}

ref(string)被用来给元素或子组件注册引用信息,引用信息会根据父组件的\$refs对象进行注册。

\begin{compactitem}
\item 如果在普通的DOM元素上使用,那么引用信息就是元素;
\item 如果在子组件上使用,那么引用信息就是组件实例。
\end{compactitem}


\begin{lstlisting}[language=JavaScript]
<!-- vm.$refs.p will be the DOM node -->
<p ref="p">hello</p>

<!-- vm.$refs.child will be the child comp instance -->
<child-comp ref="child"></child-comp>
\end{lstlisting}


当 v-for 用于元素或组件的时候,引用信息将是包含DOM节点或组件实例数组。

关于ref注册时间的重要说明如下:

\begin{compactitem}
\item ref本身是作为渲染结果被创建的,在初始渲染的时候,不能访问它们 - 它们还不存在;
\item \$refs 也不是响应式的,因此不应该试图用它在模版中做数据绑定。
\end{compactitem}



\begin{lstlisting}[language=JavaScript]

\end{lstlisting}




\begin{lstlisting}[language=JavaScript]

\end{lstlisting}




\begin{lstlisting}[language=JavaScript]

\end{lstlisting}




\begin{lstlisting}[language=JavaScript]

\end{lstlisting}




\begin{lstlisting}[language=JavaScript]

\end{lstlisting}



\chapter{slot}


slot(string)用于标记往哪个slot中插入子组件内容。




\begin{lstlisting}[language=JavaScript]

\end{lstlisting}




\begin{lstlisting}[language=JavaScript]

\end{lstlisting}




\begin{lstlisting}[language=JavaScript]

\end{lstlisting}




\begin{lstlisting}[language=JavaScript]

\end{lstlisting}




\begin{lstlisting}[language=JavaScript]

\end{lstlisting}




\begin{lstlisting}[language=JavaScript]

\end{lstlisting}

